% LaTeX source for ``Think DSP: Digital Signal Processing for Programmers''
% Copyright 2014  Allen B. Downey.

% License: Creative Commons
% Attribution-NonCommercial-ShareAlike 4.0 International
% http://creativecommons.org/licenses/by-nc-sa/4.0/
%

\documentclass[12pt]{book}
\usepackage[width=5.5in,height=8.5in,
  hmarginratio=3:2,vmarginratio=1:1]{geometry}

% for some of these packages, you might have to install
% texlive-latex-extra (in Ubuntu)

%\usepackage[T1]{fontenc}
%\usepackage{textcomp}
%\usepackage{mathpazo}
%\usepackage{pslatex}

\usepackage{url}
\usepackage{hyperref}
\usepackage{fancyhdr}
\usepackage{graphicx}
\usepackage{subfig}
\usepackage{amsmath}
\usepackage{amsthm}
%\usepackage{amssymb}
\usepackage{makeidx}
\usepackage{setspace}
\usepackage{hevea}
\usepackage{upquote}

% include this so we can compile without hyperref
% https://tex.stackexchange.com/questions/44088/when-do-i-need-to-invoke-phantomsection
\providecommand\phantomsection{}

\title{Think DSP}
\author{Allen B. Downey}

\newcommand{\thetitle}{Think DSP}
\newcommand{\thesubtitle}{Digital Signal Processing in Python}
\newcommand{\theversion}{1.1.4}

% these styles get translated in CSS for the HTML version
\newstyle{a:link}{color:black;}
\newstyle{p+p}{margin-top:1em;margin-bottom:1em}
\newstyle{img}{border:0px}

% change the arrows in the HTML version
\setlinkstext
  {\imgsrc[alt="Previous"]{back.png}}
  {\imgsrc[alt="Up"]{up.png}}
  {\imgsrc[alt="Next"]{next.png}}

\makeindex

\newif\ifplastex
\plastexfalse

\begin{document}

\frontmatter



\ifplastex
    \usepackage{localdef}
    \maketitle

\newcount\anchorcnt
\newcommand*{\Anchor}[1]{%
  \@bsphack%
    \Hy@GlobalStepCount\anchorcnt%
    \edef\@currentHref{anchor.\the\anchorcnt}%
    \Hy@raisedlink{\hyper@anchorstart{\@currentHref}\hyper@anchorend}%
    \M@gettitle{}\label{#1}%
    \@esphack%
}
\else

\newtheoremstyle{exercise}% name of the style to be used
  {\topsep}% measure of space to leave above the theorem. E.g.: 3pt
  {\topsep}% measure of space to leave below the theorem. E.g.: 3pt
  {}% name of font to use in the body of the theorem
  {}% measure of space to indent
  {\bfseries}% name of head font
  {}% punctuation between head and body
  { }% space after theorem head; " " = normal interword space
  {}% Manually specify head

\theoremstyle{exercise}
\newtheorem{exercise}{Exercise}[chapter]

\input{latexonly}

\begin{latexonly}

\renewcommand{\topfraction}{0.9}
\renewcommand{\blankpage}{\thispagestyle{empty} \quad \newpage}

% TITLE PAGES FOR LATEX VERSION

%-half title--------------------------------------------------
\thispagestyle{empty}

\begin{flushright}
\vspace*{2.0in}

\begin{spacing}{3}
{\huge \thetitle}\\
{\Large \thesubtitle}
\end{spacing}

\vspace{0.25in}

Version \theversion

\vfill

\end{flushright}

%--verso------------------------------------------------------

\blankpage
\blankpage

%--title page--------------------------------------------------
\pagebreak
\thispagestyle{empty}

\begin{flushright}
\vspace*{2.0in}

\begin{spacing}{3}
{\huge \thetitle}\\
{\Large \thesubtitle}
\end{spacing}

\vspace{0.25in}

Version \theversion

\vspace{1in}


{\Large
Allen B. Downey\\
}


\vspace{0.5in}

{\Large Green Tea Press}

{\small Needham, Massachusetts}

\vfill

\end{flushright}


%--copyright--------------------------------------------------
\pagebreak
\thispagestyle{empty}

Copyright \copyright ~2014 Allen B. Downey.


\vspace{0.2in}

\begin{flushleft}
Green Tea Press       \\
9 Washburn Ave \\
Needham MA 02492
\end{flushleft}

Permission is granted to copy, distribute, and/or modify this document
under the terms of the Creative Commons
Attribution-NonCommercial-ShareAlike 4.0 International License, which
is available at
\url{http://creativecommons.org/licenses/by-nc-sa/4.0/}.


The \LaTeX\ source for this book is available from
\url{http://think-dsp.com}.

\vspace{0.2in}

\end{latexonly}


% HTMLONLY

\begin{htmlonly}

% TITLE PAGE FOR HTML VERSION

{\Large \thetitle: \thesubtitle}

{\large Allen B. Downey}

Version \theversion

\vspace{0.25in}

Copyright 2014 Allen B. Downey

\vspace{0.25in}

Permission is granted to copy, distribute, and/or modify this document
under the terms of the Creative Commons
Attribution-NonCommercial-ShareAlike 4.0 International License,
which is available at
\url{http://creativecommons.org/licenses/by-nc-sa/4.0/}.

\setcounter{chapter}{-1}

\end{htmlonly}

\fi
% END OF THE PART WE SKIP FOR PLASTEX

\chapter{Preface}
\label{preface}
Signal processing is one of my favorite topics.  It is useful
in many areas of science and engineering, and if you understand
the fundamental ideas, it provides insight into many things
we see in the world, and especially the things we hear.

But unless you studied electrical or mechanical engineering, you
probably haven't had a chance to learn about signal processing.  The
problem is that most books (and the classes that use them) present the
material bottom-up, starting with mathematical abstractions like
phasors.  And they tend to be theoretical, with few applications and
little apparent relevance.

The premise of this book is that if you know how to program, you
can use that skill to learn other things, and have fun doing it.

With a programming-based approach, I can present the most important
ideas right away.  By the end of the first chapter, you can analyze
sound recordings and other signals, and generate new sounds.  Each
chapter introduces a new technique and an application you can
apply to real signals.  At each step you learn how to use a
technique first, and then how it works.

This approach is more practical and, I hope you'll agree, more fun.


\section{Who is this book for?}

The examples and supporting code for this book are in Python.  You
should know core Python and you should be
familiar with object-oriented features, at least using objects if not
defining your own.

If you are not already familiar with Python, you might want to start
with my other book, {\it Think Python}, which is an introduction to
Python for people who have never programmed, or Mark
Lutz's {\it Learning Python}, which might be better for people with
programming experience.

I use NumPy and SciPy extensively.  If you are familiar with them
already, that's great, but I will also explain the functions
and data structures I use.

I assume that the reader knows basic mathematics, including complex
numbers.  You don't need much calculus; if you understand the concepts
of integration and differentiation, that will do.
I use some linear algebra, but I will explain it as we
go along.


\section{Using the code}
\label{code}

The code and sound samples used in this book are available from
\url{https://github.com/AllenDowney/ThinkDSP}.  Git is a version
control system that allows you to keep track of the files that
make up a project.  A collection of files under Git's control is
called a ``repository''.  GitHub is a hosting service that provides
storage for Git repositories and a convenient web interface.
\index{repository}
\index{Git}
\index{GitHub}

The GitHub homepage for my repository provides several ways to
work with the code:

\begin{itemize}

\item You can create a copy of my repository
on GitHub by pressing the {\sf Fork} button.  If you don't already
have a GitHub account, you'll need to create one.  After forking, you'll
have your own repository on GitHub that you can use to keep track
of code you write while working on this book.  Then you can
clone the repo, which means that you copy the files
to your computer.
\index{fork}

\item Or you could clone
my repository.  You don't need a GitHub account to do this, but you
won't be able to write your changes back to GitHub.
\index{clone}

\item If you don't want to use Git at all, you can download the files
in a Zip file using the button in the lower-right corner of the
GitHub page.

\end{itemize}

All of the code is written to work in both Python 2 and Python 3
with no translation.

I developed this book using Anaconda from
Continuum Analytics, which is a free Python distribution that includes
all the packages you'll need to run the code (and lots more).
I found Anaconda easy to install.  By default it does a user-level
installation, not system-level, so you don't need administrative
privileges.  And it supports both Python 2 and Python 3.  You can
download Anaconda from \url{https://www.anaconda.com/distribution/}.
\index{Anaconda}

If you don't want to use Anaconda, you will need the following
packages:

\begin{itemize}

\item NumPy for basic numerical computation, \url{http://www.numpy.org/};
\index{NumPy}

\item SciPy for scientific computation,
  \url{http://www.scipy.org/};
\index{SciPy}

\item matplotlib for visualization, \url{http://matplotlib.org/}.
\index{matplotlib}

\end{itemize}

Although these are commonly used packages, they are not included with
all Python installations, and they can be hard to install in some
environments.  If you have trouble installing them, I
r
